\makeatletter
\usetikzlibrary{calc}
\usetikzlibrary{3d}
\usetikzlibrary{atoms.data}
\newlength\@atom@scale
\newlength\@unitcell@temp@x
\newlength\@unitcell@temp@y
\newlength\@unitcell@xx
\newlength\@unitcell@xy
\newlength\@unitcell@yx
\newlength\@unitcell@yy
\newlength\@unitcell@zx
\newlength\@unitcell@zy
\newlength\@unitcell@xlen
\newlength\@unitcell@ylen
\newlength\@unitcell@zlen
\setlength\@atom@scale{0.25in}
\def\useatomscheme#1{\usetikzlibrary{atoms.style.#1}}
\def\extractxy#1,#2\relax{%
    \pgfmathsetlength{\@unitcell@temp@x}{#1}%
    \pgfmathsetlength{\@unitcell@temp@y}{#2}%
}
\pgfkeys{
    /unitcell/.cd,
    color scheme/.initial={jmol},
    name/.initial={},
    atom scale/.initial=1,
    x/.code={
        \extractxy#1\relax
        \setlength\@unitcell@xx{\@unitcell@temp@x}
        \setlength\@unitcell@xy{\@unitcell@temp@y}
        \pgfmathsetlength\@unitcell@xlen{%
            sqrt(\@unitcell@xx*\@unitcell@xx+\@unitcell@xy*\@unitcell@xy)%
        }
    },
    y/.code={
        \extractxy#1\relax
        \setlength\@unitcell@yx{\@unitcell@temp@x}
        \setlength\@unitcell@yy{\@unitcell@temp@y}
        \pgfmathsetlength\@unitcell@ylen{%
            sqrt(\@unitcell@yx*\@unitcell@yx+\@unitcell@yy*\@unitcell@yy)%
        }
    },
    z/.code={
        \extractxy#1\relax
        \setlength\@unitcell@zx{\@unitcell@temp@x}
        \setlength\@unitcell@zy{\@unitcell@temp@y}
        \pgfmathsetlength\@unitcell@zlen{%
            sqrt(\@unitcell@zx*\@unitcell@zx+\@unitcell@zy*\@unitcell@zy)%
        }
    },
    x/.initial={(1.00cm,0.00cm)},
    y/.initial={(0.20cm,0.05cm)},
    z/.initial={(0.00cm,1.00cm)},
    scale/.initial=1,
}
\tikzset{
    /unitcell/atom/.style={
        thin,
        circle,
        draw=black,
        fill=atom-#1,
        /unitcell/atom-style-#1,
    },
    /unitcell/atom/.default={X},
    /unitcell/edge/.style={
        thin,
        draw=black,
        shorten >= 0.375\@atom@scale,
        shorten <= 0.375\@atom@scale
    },
    /unitcell/bond/.style={
        ultra thin,
        draw=black,
        shorten >= 0.5\@atom@scale,
        shorten <= 0.5\@atom@scale,
        double=atom-#1,
        /unitcell/atom-style-#1,
        minimum size=0pt,
        double distance=0.125\@atom@scale,
        cap=round,
    },
    /unitcell/bond >/.style={/unitcell/bond=#1,shorten <=0},
    /unitcell/bond </.style={/unitcell/bond=#1,shorten >=0},
}

\def\unitcell[#1]{
    \pgfkeys{/unitcell/.cd,#1}
    \pgfkeysgetvalue{/unitcell/color scheme}\@unitcell@scheme
    \pgfkeysgetvalue{/unitcell/atom scale}\@unitcell@atom@scale
    \pgfkeysgetvalue{/unitcell/name}\@unitcell@name
    \pgfkeysgetvalue{/unitcell/scale}\@unitcell@scale
    \useatomscheme{\@unitcell@scheme}
    \begin{scope}[
        x={(\@unitcell@xx,\@unitcell@xy)},
        y={(\@unitcell@yx,\@unitcell@yy)},
        z={(\@unitcell@zx,\@unitcell@zy)},
        scale=\@unitcell@scale
    ]
    \pgfmathsetlength\@atom@scale{
        \@unitcell@scale * \@unitcell@xlen * \@unitcell@atom@scale
    }
    % \coordinate (\@unitcell@name\space origin) at (0,0,0);
}
\def\endunitcell{
    \end{scope}
}
\def\addatom at site (#1,#2,#3)#4;{
    \node[/unitcell/atom={#4}] at (#1,#2,#3) {};
}
\def\addedge from (#1,#2,#3) to (#4,#5,#6);{
    \draw[/unitcell/edge] (#1,#2,#3) -- (#4,#5,#6);
}
\def\addbond from (#1,#2,#3) #4 to (#5,#6,#7) #8;{
    \edef\@atom@a{#4}\edef\@atom@b{#8}
    \ifx\@atom@a\@atom@b
    \draw[/unitcell/bond = {#4}] (#1,#2,#3) -- (#5,#6,#7);
    \else
    \draw[/unitcell/bond <= {#4}] (#1,#2,#3) -- ($(#1,#2,#3)!0.5!(#5,#6,#7)$);
    \draw[/unitcell/bond >= {#8}] ($(#1,#2,#3)!0.5!(#5,#6,#7)$) -- (#5,#6,#7);
    \fi
}

\makeatother